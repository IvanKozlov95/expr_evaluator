\documentclass{42-en}


%******************************************************************************%
%                                                                              %
%                                    Header                                    %
%                                                                              %
%******************************************************************************%
\begin{document}



                           \title{Interpreter101}
                          \subtitle{Create your own language instead of learning one!}
                       \member{Ivan Kozlov}{ikozlov@student.42.us.org}
                        \member{Kai Drumm}{kai@42.us.org}

\summary {
  Summary: The aim of this project is to code a simple interpreter. (Don't worry. It's easier then it seams.)
}

\maketitle

\tableofcontents


%******************************************************************************%
%                                                                              %
%                                  Foreword                                    %
%                                                                              %
%******************************************************************************%
\chapter{Foreword}

    In case you missed Geography in school or just want to learn some exciting facts about the world you live in - here are 20 facts about Poland. Read away...\\

            \begin{enumerate}\itemsep7pt
                \item The most popular dog’s name in Poland is “Burek”, which is the Polish word for a brown-grey colour.
                \item In Poland, people generally peel bananas from the blossom end and not the stem end.
                \item Poland shares its borders with no less than seven countries: Russia, Lithuania, Belarus, Slovakia, Ukraine, the Czech Republic and Germany.
                \item Poland has had many capital cities in its time. These have included Gniezno, Poznan, Krakow and Warsaw. Lublin has been the capital twice – after both World Wars.
                \item Geographically, Poland is not actually in Eastern Europe – it is in fact in the very centre of Europe(this inaccuracy personally drives me crazy)
                \item In Poland, one’s “Name Day” – imieniny – is considered a far more important occasion than one’s birthday.
                \item Poland’s national symbol is the white-tailed eagle.(get that US)
                \item The name “Poland” originates from the name of the tribe Polanie. This tribe used to inhabit the Western part of what we now call Poland, and originally meant “people living in open fields”.
                \item The most “World’s Strongest Man” winners are from Poland.
                \item People of Polish descent have won 17 noble prizes, including four for peace and five for literature.
                \item One third of Poland is covered with forest, with 50% of the land is dedicated to farming. The country contains 9,300 lakes, 23 National Parks and one desert.
                \item Rysy, in the Tatra Mountains, is the highest point in Poland, at a height of 2,499 m.
                \item Most Poles are Roman Catholics.
                \item Roman Catholicism is so popular that Poland has a TV channel dedicated to the Pope.
                \item In 1989, Poland held its first free elections in more than 40 years.
                \item Bigos is the country’s most popular traditional dish. It is a type of stew made from Polish sauerkraut, fresh cabbage, different types of meat, sausage, prunes, dried mushrooms, onions and spices. This is cooked over several days and served with potatoes and bread.
                \item Marzanna is a Polish tradition where people weave straw dolls, which they then decorate with ribbons. These dolls represent winter, so when the snow starts to melt the Marzanna dolls are thrown into a river, symbolizing the ‘killing’ of winter.
                \item The Polish are generally well educated, with 90% of Poles having completed at least a secondary education.
                \item The Polish born astronomer Nicolaus Copernicus is considered to be the first person to propose the theory that the earth was not the centre of the universe.
                \item In Poland, pizza bases are not topped with Napolitana or a tomato-based sauce. These are generally served separately and resemble what we would consider to be ketchup.\\
            \end{enumerate}

        \warn{
            Usually foreword has no connection to subject whatsoever. But not in this case. What is so polish about this topic? Can you read these facts in reverse?
        }

%******************************************************************************%
%                                                                              %
%                                 Introduction                                 %
%                                                                              %
%******************************************************************************%
\chapter{Introduction}

    Programming languages were around since 1942. First language was Plankalkül
    or  \texttt{Plan Calculus}. It was a simple(not really) programming language
    which mostly focused on algebra and made calculations a lot easier.
    
    Since then languages have developed to be able to do complex things(such as
    the operation system you are using right now). But computers haven't changed much.
    There is still a need to translate the language to commands a machine can understand.
    
    There are two types of languages: \textbf{compiling} and \textbf{interpreted}.
    We will be focusing on the second one for several reasons(you don't want to learn
    assembly, do you?)

%******************************************************************************%
%                                                                              %
%                                  Goals                                       %
%                                                                              %
%******************************************************************************%
\chapter{Goals}
    
    The goal of this project is to introduce you to inner workings of
    programming languages. Sounds fun? No? Too bad.
    
    In this project you will create you own language. It won't do much, but you
    have to start somewhere, right? At the very end your language will calculate
    a mathematical expression, which may include variables.
    
    You also will get familiar with the \texttt{Stack} data structure. 
    This is very a common way to organize an array of elements
    used both in real life and in software development. There are many real life
    examples of a stack. Consider the simple example of plates stacked over one 
    another in the restaurant. The plate which is at the top is the first one to be 
    removed, i.e. the plate which has been placed at the bottom most position 
    remains in the stack for the longest period of time. 
    So, it can be simply seen to follow LIFO/FILO order.
    
    \hint{
        LIFO stands for \texttt{Last In First Out}
    }
    
    For the reasons of common sense and countless others, stack structure t
    must be implemented using OOP.
    
    \hint{
        'P' in OOP stands for programming :)
    }

    The stack is not much use by itself, but it can be very useful tool to solve certain
    problems.

    Through this project, you will use a stack to convert an expression to a form
    that a machine will understand and then calculate. Simple? We'll see about that...


%******************************************************************************%
%                                                                              %
%                             General instructions                             %
%                                                                              %
%******************************************************************************%
\chapter{General instructions}

    \begin{itemize}\itemsep1pt
        \item You will use Python 3 in this project(New is always better. \href{https://en.wikipedia.org/wiki/Barney_Stinson}{True story})
        \item You are not allowed use any modules except \texttt{sys}
        \item You are free to organize and name your files as you wish,
        but within the constraints listed here.
        \item Besides the files you need, you must turn in three files named
        \texttt{stack.py}, \texttt{interpreter.py} and \texttt{main.py}
        \item \texttt{stack.py} must have an implementation of class Stack
        \item \texttt{interpreter.py} is the main class of your program. It will get the 
        code as text, parse and execute it
        \item \texttt{main.py} must be a main executable, which will calculate an 
        expression and print the result on stdout
        \item All files must be placed at root of the folder
        \item In no way you program should exit unexpectedly. You are allowed to use
        custom exception errors.
        \item May the Force be with you. Always.
    \end{itemize}

%******************************************************************************%
%                                                                              %
%                             Mandatory part                                   %
%                                                                              %
%******************************************************************************%
\chapter{Mandatory part}

    Your language will have simple and straightforward syntax. Here is the example
    
\begin{42ccode}
a = 1
b = 2
c = 3
a + b + c
\end{42ccode}

    Your test file will contain \texttt{n} lines of code. The first \texttt{n - 1} lines
    of this file will be variables. These lines should follow simple syntax:
\begin{42ccode}
name = value
\end{42ccode}
    Where \texttt{name} is a letter of Latin alphabet and \texttt{value} is an integer.
    
    The last line must be an mathematical expression which may contain variables, integers, and math operations. Just like in normal mathematical expression. No tricks.
    
    Your program must support following operations:
    \begin{itemize}\itemsep1pt
        \item addition - represented by \texttt{+}
        \item subtraction - represented by \texttt{-}
        \item modulo - represented by \texttt{\%}
        \item division - represented by \texttt{/}
        \item multiplication - represented by \texttt{*}
        \item power - represented by \texttt{\^}
        \item don't forget about parenthesis :) 
    \end{itemize}

    \hint {
              Think about order of operations. Multiplication always goes before addition
              but parenthesis can change the order.
            }

    \section{Small steps in the right direction}
    
    Although creating an interpreter might seem overwhelming and it is. You can start by creating class \textbf{Stack}. Trust me you will need it at the end.
    
    \texttt{stack.py} must have an implementation of the class \texttt{Stack}
    with following methods:

            \begin{itemize}\itemsep1pt
                \item push - pushes value on top of the stack
                \item pop - returns value of top element of the stack and removes it
                \item peek - returns value of top element of the stack \textbf{without} removing it
                \item isEmpty - returns \texttt{true} if stack is empty and \texttt{false} otherwise
            \end{itemize}
    Here is a skeleton for you: 
\begin{42pycode}
class Stack:
    def __init__(self):
        pass
    
    def push(self):
        pass
    
    def pop(self):
        pass

    def peek(self):
        pass

    def isEmpty(self):
        pass
\end{42pycode}

    \section{Interpreter}

    The interpreter is the main class of your program. It will parse your code line by 
    line, store all variables and then execute the last line as a mathematical 
    expression. The variables will be replaced by their values.

    \section{Program entry}
    
        The program will take the 1st argument that is passed. This argument represents a 
        path to the file with your program code. In case there are too many arguments(or 
        none) your program 
        should display its usage:
        \newline
        \begin{42console}
> python3 main.py | cat -e
Usage: python3 main.py [path_to_file]
\end{42console}
        \info{
                  All operators and operands \textbf{must} be separated by spaces. The only exceptions are negative values like \texttt{-2}.
                }
                
        \info{
                  The program will be correct and \textbf{will not} contain any errors.
            }
        Here is an example of how you program should work:
        
        \begin{42console}
> cat ../examples/test1
a = 1
b = 2
c = 3
a + b + c

> python main.py ../examples/test1
6
\end{42console}


%******************************************************************************%
%                                                                              %
%                                 Bonus part                                   %
%                                                                              %
%******************************************************************************%
\chapter{Bonus part}

    We all like bonuses, don't we? Evaluating an expression is just the first step.
    You can evaluate everything! Python, for example, is an interpreted language.
    That means that a special program called an interpreter runs the code and evaluates your commands step by step. You can think about it as a complex expression with support for more operations than mathematical. Here are some ideas to give you a chance to improve your program.
     \begin{description}\itemsep3pt
        \item [Errors:] people aren't perfect but your program can be! Make it check if
        the input is correct. It can give an error message to a user, explaining what is wrong.
        \textcolor{blue}{You} \textcolor{green}{can} \textcolor{orange}{add} \textcolor{red}{colors}!
                \begin{42console}
> ./main.py error_file | cat -e
[ERROR] Variable 'f' is not defined
\end{42console}
        \item [More operations:] Basic math operations are fun but not \textsc{superfun}.
        Add a new operation to your program. Like binary operators, or absolute values.\\
    \end{description}
    

%******************************************************************************%
%                                                                              %
%                           Turn-in and peer-evaluation                        %
%                                                                              %
%******************************************************************************%
\chapter{Turn-in and peer-evaluation}

    Submit your work on your \textbf{git} repository as usual. Only the work on your 
    repository will be graded.

%******************************************************************************%
\end{document}
